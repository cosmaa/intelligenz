% Encoding: UTF-8
\newpage
\thispagestyle{plain}
\clearpage
\hfuzz=12pt       % suppress warnings due to extenstion onto page margins

\vspace{0.3cm}
\textbf{Aufgabenstellung}

\vspace{0.3cm}
Die schriftliche Ausarbeitung soll folgenden Inhalt haben:
\begin{itemize}
\item Für "Suchen", "Lernen" und "Verarbeitung von Sequenzen" jeweils eine
Stichwortliste für die Konzepte, die Sie sich erarbeitet haben.
(Konzepte sind z.B. Algorithmen und ihre Eigenschaften, Formeln,
Begriffe mit ihrer Bedeutung.)
\item  Maximal fünf Seiten zum Thema "Intelligenz". Schriftgröße und
Formatierung sollten sich an der von Bachelorarbeiten orientieren, so
ganz genau messen wir nicht nach. Bitte keine Definitionen für
Intelligenz. Beziehen Sie sich auf eine seriöse Quelle (oder mehrere),
die sich mit der Frage auseinandersetzt, was mit Künstlicher Intelligenz
möglich sein wird und weshalb. Oder: Wie sich natürliche und künstliche
Intelligenz unterscheiden. Oder: Was nicht möglich sein wird. Die Quelle
darf auch aus anderen Disziplinen stammen: Philosophie, Psychologie,
Neurobiologie,... Fassen Sie die Aussagen und Argumentationen der Quelle
zusammen und setzen Sie sich damit dann kritisch auseinander. Wo stimmen
Sie überein, wo nicht. Dies bitte gut begründen. (Falls Sie eine etwas
andersartige Fragestellung zur Intelligenz bearbeiten und nicht sicher
sind, ob das passt, fragen Sie gern nach.)
\item Ca. eine halbe Seite zu Ihrer Verantwortung als Informatiker. Denken
Sie gründlich darüber nach, was Sie persönlich tun möchten und was
nicht. Wie soll die Welt sein, in der Sie leben, und was können Sie dazu
beitragen? (Das Ergebnis Ihrer Überlegungen wird natürlich nicht
bewertet. Es soll aber sichtbar sein, dass Sie sich mit der Frage
auseinander gesetzt haben.)

\end{itemize}
