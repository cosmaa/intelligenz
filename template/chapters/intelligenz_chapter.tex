% Encoding: UTF-8

\section{Künstliche Intelligenz}
Das Buch “21 Lektionen für das 21.Jahrhundert” \footnote{\ Harari2019 3. Ausgabe}\cite{harari201921}, widmet sich den technologischen und politischen Herausforderungen und anderen drängenden Fragen der heutigen Zeit. Der Autor Yuval Noah Harari ist Professor für Geschichte an der Hebrew University in Jerusalem. Harari gibt im Abschnitt “Künstliche Intelligenz und natürliche Dummheit”\footnote{\  Harari2019 Seite 124-129 } einen Ausblick auf die möglichen Auswirkungen von künstlicher Intelligenz auf die natürliche Intelligenz des Menschen.

\begin{adjustbox}{minipage=0.95\textwidth,margin=0pt \smallskipamount,right}
Die folgende Sektion fasst den Text in seinen wesentlichen Zügen zusammen. Anschließend befasse ich mich mit ausgewählten Argumenten des Autors.
\end{adjustbox}


\subsection{Künstliche Intelligenz und natürliche Dummheit}

Der erste Teil dieses Kapitels beschäftigt sich mit menschlicher und auch künstlicher Intelligenz (KI) im Zusammenhang mit Bewusstsein. Anders als dystopische Zukunftsvisionen aus Medien nämlich nahelegen, gibt es zur Zeit keinen Grund anzunehmen, das es Maschinen überhaupt möglich ist ein Bewusstsein zu entwickeln. Somit sind Maschinen auch nicht in der Lage Menschen aus einem eigenen Bewusstsein heraus zu beeinflussen, da sie nur auf Algorithmen aufbauen. Das Bewusstsein an sich nämlich ist ein biochemischer Mechanismus der es dem Menschen erlaubt Gefühle zu empfinden. Die Intelligenz allerdings ist nichts weiter als die Begabung Probleme zu lösen. Da bei Menschen jedoch Bewusstsein und Intelligenz Hand in Hand gehen, wenn es um das Lösen von Problemen geht werden diese Begriffe oft synonym verstanden. Und genau dort liegt der Unterschied, da eine Maschine zum Lösen menschlicher Problemstellung nur analytische Intelligenz benutzt ohne dabei Empfindungen zu haben. 
Ferner wird darauf eingegangen, dass nicht gesagt ist, das KI niemals in der Lage sein wird Bewusstsein zu erlangen. Das liegt vor allem auch daran, dass der aktuelle Forschungsstand zum Bewusstsein des Menschen bei Weitem nicht genügt um hierüber eindeutige, qualifizierte Aussagen machen zu können. Es gäbe laut Harari drei Möglichkeiten, wie Bewusstsein schlussendlich funktionieren könne. Entweder Bewusstsein ist direkt mit organischer Biochemie verbunden und es wird niemals bewusste KI geben können. Die zweite Möglichkeit wäre, dass Bewusstsein nur aus Intelligenz ergibt und KI damit früher oder später zwangsläufig Bewusstsein erlangt. Oder es besteht weder zwischen Bewusstsein und organischer Biochemie noch zwischen Bewusstsein und hoher Intelligenz ein Zusammenhang. Somit könnten Computer eine Bewusstsein, was aber keine notwendiger Vorgang wäre. Dieser Mangel an Wissen lässt eine gezielte Entwicklung bewusster KI durch Menschen unglaubwürdig erscheinen. 
Nun ist es gerade diese Kombination aus Unwissenheit über das Bewusstsein und der Entwicklung von KI, welche emotionslos und  analytisch vorgeht, die Gefahren in sich birgt. So wären Computer besser in der Lage Menschen zu in kalkulierte Bahnen zu lenken, etwa im politischen Sinne. Denn eine KI ist auf analytischer Basis gezielt in der Lage menschliche Gefühle zu beeinflussen. Ein Beispiel hierfür ist die Manipulation von Wählern in dem deren Daten analysiert werden und bestehende Vorurteile jener genutzt werden.
Ebenso, wie in das Entwickeln von Algorithmen und das Erschaffen künstlicher Intelligenz, sollte daher in die Ergründung des Bewusstseins investiert werden. Geschieht dies nicht besteht die Gefahr, dass Menschen zu bloßen Datenproduzenten verfallen und nicht mehr in der Lage sind das eigene Potential auszuschöpfen. Dieses würde auf kurz oder lang zur Degenerierung des Menschen durch den Einsatz und den Umgang mit KI führen, welche eigentlich nur ein Werkzeug sein sollte. Am Ende das Kapitels geht beschreibt Harari dann eine eigene Zukunftsdystopie. So kann der unbewusste Umgang mit KI in eine Zukunft führen, in der eine sehr kleine Elite alles in den Händen hält, während sich der ganze Rest der Menschheit nicht mehr annähernd in der Lage sein wird das eigene Potential zu nutzen oder auch nur im Enferntesten einschätzen zu können. \\

Harari spricht in diesem Kapitel, meiner Meinung nach einige wichtige Punkte an, die bei der Entwicklung und dem Gebrauch von KI aktuell noch keine zentrale Rolle spielen. Ich bin nicht sicher ob seine drei Möglichkeiten des Zusammenspiels aus Intelligenz, biochemischen Prozessen und Bewusstsein nicht zu stark vereinfacht ist. Es könnte meiner Ansicht nach nötig sein den Begriff des Bewusstseins weiter zu fassen, sollte man diese jemals wissenschaftlich besser verstehen. Ich denke allerdings Hararis Kritik an dem unbewussten Umgang mit KI und gleichzeitig dem  Vernachlässigen des bewussten Reflektieren darüber ist gerechtfertigt. Ob die von ihm beschriebene Dystopie nun so zustande kommt oder nicht sei hierbei außen vor gelassen. Wie allerdings schon heutzutage ein komplett unreflektierter Umgang mit innovativer Technik an den Tag gelegt wird lässt sich meiner Ansicht nach schlichtweg nicht von der Hand weisen. Beispielsweise braucht niemand jedes Jahr ein neues Smartphone bei dem bestenfalls die Kamera etwas besser ist als beim Vorgängermodell. In diesem Fall stellt sich außerdem die Frage, in wie Fern es innovativ sein kann, Jahr für Jahr die selbe Technik mit ganz leichten Optimierungen herauszubringen. Auch das man sich durch das ständige Benutzen und mit sich herumtragen eines Smartphones zur 'Datensammelmaschine' macht und das durch die jährlich neu aufkommenden Modelle ein unermessliche Verschwendung von Rohstoffen und Müllbelastung sind wenn überhaupt nebensächliche gesellschaftliche Themen. Da also Menschen jetzt schon komplett unreflektiert mit technischer Innovation umgehen, gebe ich Harari in dem Punkt recht, dass sich mit der Weiterentwicklung der KI ebenfalls das menschliche Bewusstsein besser ergründet werden muss um angemessen mit technischer Innovation umgehen zu können. 


\subsubsection{KI löst Probleme}

Egal in welchen Bereich man Blick, die Distanz zwischen nützlich und gefährlich ist gering. Nützlich in dem Sinn dass KI  menschliche Aufgaben oder Probleme  löst und bewältigt. So zum Beispiel hilft KI den Menschen im Bereich der Medizin Krankheiten effizienter zu diagnostizieren, Medikamente zu entwickeln, Behandlungen zu personalisieren und Gene zu verändern. Je mehr man medizinische Daten digitalisiert und vereinheitlicht, desto mehr kann KI dazu genutzt werden, relevante Muster zu finden - Muster, mit denen komplexe Analyseprozessen Ärzte präziser, schneller und  kostengünstiger Entscheidungen treffen können.\cite{Schmitt19Medizin}\ Der Patienten hat so dann auch eine höhere Lebenserwartung. Kritisch ist es dann eher nicht dass Menschen dadurch trotzdem sterben - die Wahrscheinlichkeit ist sehr viel geringer.
Es wird gefährlich wenn Maschinen individuelle Therapien vorschlagen und der Arzt gehemmt ist, diese Therapien vom intelligentesten Mediziner der Welt infrage zu stellen und davon abzuweichen.
Entweder degeneriert der Mensch dadurch weil er die Fähigkeit verliert sich mit Problemen auseinander zu setzten oder er bekommt die Chance sich mit neuen Aufgaben zu beschäftigen, zB. wie vermeiden wir Krankheiten.

\subsubsection{KI und Bewusstsein}

Die Interpretationen und Anschauungsweisen sind kontrovers, es wird viel diskutiert versucht zu differenzieren der Begrifflichkeiten. Wie Harari aber schon sagt ist das Wissen über das Bewusstsein noch sehr begrenzt. Folglich entwickeln sich Thesen wie diese:\\
Das Bewusstsein sei beim Menschen an eine klare Auswahl von Einzelneuronen gebunden, auf die man sich konzentrieren sollte, um Bewusstsein künstlich zu erzeugen.\cite{RoepkeKIT19InterviewWendland} Dem gegenübergestellt wird - das, was man Bewusstsein nennt ist insgesamt ein soziales Konstrukt, das nicht eindeutig einem Kopf, Körper oder Leib zugerechnet werden kann. Aus diesen Positionen heraus lägen KI-Bewusstseinsüberlegungen im Bereich der Fantasie.\cite{RoepkeKIT19InterviewWendland} Eher technisch sind diese Sichtweisen- Mit steigender Komplexität stellt sich ein solches Bewusstsein irgendwann automatisch ein. Es erscheine nicht aus dem Nichts, sondern aus der Komplexität der Algorithmen oder der Anzahl der Speicherzellen und des Vernetztheitsgrades. Dies Entspräche den Gehirnzellen eines Menschen, was sich auf der Ebene künstlichen neuronalen Netzen widerspiegelt. Würde man also viele Neuronen nachbauen, gäbe man diesen eine neue Bedeutung eines eigenen Bewusstseins. \cite{HartliebKIT19InterviewWendland}

Nicht nur Harari schwimmt in seiner Ansicht zur Bewusstseinsentwicklung von Maschinen. Diese Debatte reicht weit zurück und hat bis heute keine klare Antwort gefunden. Es existieren kontrastierte Ansichten. 2019 gab es einen umfangreichen Dialog zu dem Thema, organisiert von der KIT (Karlsruher Institut für Technologie) die mit Wissenschaftlern, Philosophen, Soziologen, Science-Fiction Roman Autoren und anderen Kulturkreisen im Austausch standen um eine Einschätzung der Lage zu geben und die Begriffe klar zu deffinieren. Dieses Projekt „Abklärung des Verdachts aufsteigenden Bewusstseins in der Künstlichen Intelligenz“\cite{Wendland19VerdachtaufBWS} wird vom Bundesministerium für Bildung und Forschung gefördert. Hier wird dennoch betont dass keine endgültige Antwort greifbar ist. Hier sieht man dass sich Menschen angetrieben von dem Gefühl, es könnte ein Problem entstehen, aktiv und gemeinschaftlich um präventive Maßnahmen kümmern.


\subsubsection{KI schafft Ungleichheit}

Harari betrachte die Störung des liberalen Systems und warnt vor Ungleichheit. Doch auch hier gibt es Zwei Sichtweisen Ungleichheit schafft Gerechtigkeit und Gleichheit schafft Ungerechtigkei (Abb.1) \footnote{ \ https://www.tollabea.de/warum-gleichbehandlung-von-kindern-einfach-eine-doofe-idee-ist/}.

\includenamedimage[0.5]{Gleich.jpg}{ Ungleichheit und Gleichheit}

Man schafft Gleichheit mit dem Hilfsmittel Maschine. Zum Beispiel half das Kommunikationssystem von Stephen Hawking, sich in der Gesellschaft mitzuteilen und sogar sie aktiv zu gestalten. Der physischen Ungleichheit wird durch Software und Hardware entgegen gesteuert. Ungleichheit entsteht aber genau dann wenn nicht auch jedern andere Mensch mit der gleichen Erkrankung diese Maßnahmen erhält.

Schwierig sind Meinungsbildende Algorithmen. Das sind algorithmische Empfehlungssysteme die – unter Umständen – Filterblasen und Echokammern in sozialen Medien erzeugen können. Dabei ist eine Filterblase das Phänomen, dass man von Algorithmen hauptsächlich solche Themen immer wieder vorgeschlagen bekommen, die man schon mag. Als Echokammern bezeichnet man Freundesgruppen, die hauptsächlich aus Leuten mit ähnlicher Meinung bestehen. Es ist noch unklar wie sich dieser Eingriff in soziale Strukturen auf die Menschen auswirkt. Doch Laut Zweig kann mit Hilfe dieser Art von Algorithmen die Meinung der Menschheit manipuliert werden\cite{Zweig17Meinungsbildung}. Sie schaffen Ungleichheit weil Informationen selektiv für jeden einzelnen bereit gestellt wird. Explizit heißt das, wenn der Mensch sich nicht aktiv um eine Vielzahl an Informationen bemüht, also aus seiner Blase schaut, werden sich wohl Menschengruppen bilden die verkümmern, an der Monotonie ihrer Blase. Menschen wachsen an Konflikten die ihnen begegnen doch wenn diese Konflikte gar nicht erst entstehen, muss auch kein Problem gelöst werden.

\begin{adjustbox}{minipage=0.95\textwidth,margin=0pt \smallskipamount,right}
Persönliche Haltung:
Ich finde wir brauchen noch viel mehr Diskussion darüber, an welchen Stellen künstliche Intelligenz unseren unreflektierten Neigungen inklusive unserer natürlichen Dummheit dient. Es sollte gesetzlich geregelt sein welche sinnvollen gesellschaftliche Aufgaben der Einsatz von KI benötigt. Und man sollte darüber nachdenken welche Schritte unternommen werden können, um ein bisschen mehr in Richtung von Intelligenz, ob natürlich oder künstlich, zu gehen. 
\end{adjustbox}


\subsection{Zur persönlichen Verantwortung als Software-Entwicklerin}

In meiner Echokammer kommt häufiger die Aussage "Die Technologien lassen die Menschen verdummen". Ich bin da im Zwiespalt.
Ich denke wenn man Informatik so einsetzt, dass sie uns Dinge erlaubt die wir vorher nicht konnten, dann ist sie nicht schlecht und macht uns auch nicht dumm. Das ist auch meine primäre Motivation meinen Mitmenschen die Arbeit zu erleichtern und sogar abzunehmen, da ja nun mal nicht jede Arbeit glücklich macht.
Ich bin der Ansicht Dumm macht uns die Informatik, wenn sie uns Dinge abnimmt, die wir vorher selbst konnten.
Zum Beispiel wenn ich immer mit Navigationsgerät fahre, dann brauche ich keine Navigationsfähigkeiten mehr und kann entsprechend kaum noch Strassenschilder in geeignetem Tempo erfassen. Oder Wenn ich alle Telefonnummern im Handy speichere, dann muss ich nicht mal mehr die Telefonnummer meiner Mutti auswendig kennen, das Gedächtnis wird also nicht mehr so stark benötigt. Eigentlich gilt das ja noch nicht mal nur für die Informatik. Wenn ich alles in einen Terminkalender aus Papier eintrage, dann muss ich mir nicht mehr dutzende Termine merken können. Einzeln wären diese Dinge nicht weiter schlimm. Es wird aber mittlerweile so viel frühere Hirnarbeit von Computern übernommen, dass bestimmte Hirnfunktionen degenerieren.
Aber eben, das betrifft nur ein Teil der Informatik. Ein anderer Teil hat darauf keinen Einfluss.


