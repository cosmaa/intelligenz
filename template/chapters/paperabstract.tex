% Encoding: UTF-8
\newpage
\thispagestyle{plain}
\clearpage
\hfuzz=12pt       % suppress warnings due to extenstion onto page margins

\vspace{0.3cm}
\textbf{Kurzzusammenfassung}

\vspace{0.3cm}
Diese Ausarbeitung gibt im ersten Teil einen Ausblick auf die im Referat erklärten Kernkonzepte von Monte-Carlo-Tree-Search Algorithmus,   Generative Adversarial Networks und  Long-Short Term Memorys.\newline 
Der zweiten Teil befasst sich mit der Fragestellung -  Wie unterscheiden sich natürliche und künstliche
Intelligenz. Zentral hier für ist das Buch ”21 Lektionen des 21. Jahrhundert”\cite{harari201921}, insbesondere der Abschnitt “Künstliche Intelligenz und natürlich Dummheit”\footnote{Harari 2019, Teil I - Die technologische Herausforderung - Abschnitt Freiheit - Seite 124/129} von Yuval Noah Harari.\newline
Eine persönliche Stellungnahme zur Verantwortung als Software-Entwicklerin schließt die Ausarbeitung ab.
